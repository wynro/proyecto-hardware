\documentclass[12pt,letterpaper]{article}
\usepackage[spanish]{babel}
\usepackage[utf8]{inputenc}
\usepackage[right=2cm,left=3cm,top=2cm,bottom=2cm,headsep=0cm,footskip=0.5cm]{geometry}
\usepackage[pdftex]{graphicx}
\usepackage{hyperref}
\usepackage{makeidx}
\usepackage{wynroTitle}
% \usepackage{verse}
% \newcommand{\attrib}[1]{%
% \nopagebreak{\raggedleft\footnotesize #1\par}}
% \renewcommand{\poemtitlefont}{\normalfont\large\itshape\centering}

\makeindex
\pagenumbering{arabic}

\logo{logoUZ.png}
\subject{Proyecto Hardware\\Trabajo de la asignatura}
\title{Implementación de un juego sudoku\\\vspace{1cm}Resumen ejecutivo}
\author{Guillermo Robles González - NIP: 604409}
\supervisor{Javier Resano Ezcaray (coordinador)\\
María Villarroya Gaudó\\
Enrique Torres Moreno\\
Jesús Alastruey Benedé\\
Darío Suárez Gracia}

% It's Dangerous to Go Alone! Take This
\begin{document}
% Primera pagina (Titulo)
\maketitle

El trabajo mandado, consistente en el diseño y la implementación de un
sencillo juego de sudoku, fue completado con éxito.

El juego cumple con todas las características pedidas, ademas de un
pequeño conjunto de características extra que, pese a no pertenecer al
conjunto pedido, mejoran la jugabilidad y comodidad del usuario,
conformando un proyecto sólido.

Algunas de las características extra que no aparecían en los
requisitos son:
\begin{itemize}
  \item Adición de pantalla de título y creditos
  \item Uso de la librería Bmp dada por los profesores para la adición
    de sencillas imágenes al apartado gráfico
  \item Generalización de funciones. Por ejemplo, si se diera el caso
    de tener que llevar el proyecto a una pantalla con el doble de
    resolución, cambiando 4 números y la fuente el proyecto puede
    ejecutarse siin problemas.
  \item Uso de cursor para la selección de cuadros y números, que
    permiten el casi total abandono del 8 segmentos.
\end{itemize}

En conclusión, el proyecto se ha finalizado satisfactoriamente,
alcanzando un nivel aceptable tanto en los objetivos marcados por el
profesorado como en los objetivos personales marcados por el alumno.
\end{document}
