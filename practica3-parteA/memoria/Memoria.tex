\documentclass[12pt,letterpaper]{article}
\usepackage[spanish]{babel}
\usepackage[utf8]{inputenc}
\usepackage[right=2cm,left=3cm,top=2cm,bottom=2cm,headsep=0cm,footskip=0.5cm]{geometry}
\usepackage[pdftex]{graphicx}
\usepackage{hyperref}
\usepackage{makeidx}
\usepackage{wynroTitle}
% \usepackage{verse}
% \newcommand{\attrib}[1]{%
% \nopagebreak{\raggedleft\footnotesize #1\par}}
% \renewcommand{\poemtitlefont}{\normalfont\large\itshape\centering}

\makeindex
\pagenumbering{arabic}

\logo{logoUZ.png}
\subject{Proyecto Hardware\\Trabajo de la asignatura}
\title{Implementación de un juego sudoku} %TODO: Cambiar
\author{Guillermo Robles González - NIP: 604409}
\supervisor{Javier Resano Ezcaray (coordinador)\\
María Villarroya Gaudó\\
Enrique Torres Moreno\\
Jesús Alastruey Benedé\\
Darío Suárez Gracia}

% It's Dangerous to Go Alone! Take This
\begin{document}
% Primera pagina (Titulo)
\maketitle
% Segunda pagina (Indice)
\tableofcontents
% El resto
\section{Resumen}
% Es un apartado fundamental. Es lo primero que se lee y muchas*
% veces lo único que lee. Es la síntesis de todo el trabajo
% realizado, qué, cómo y por qué hemos hecho el trabajo. Debe ser
% auto contenido y debemos esbozar nuestras conclusiones.
% [*] todas

\section{Introducción}
% Enmarca y sitúa el trabajo a realizar.

\section{Objetivos}
% Explica qué se quiere conseguir.

\section{Metodología}
% Describe los pasos realizados para llegar hasta los
% resultados. Todas aquellas decisiones de diseño tomadas en el
% proceso deben incluirse.En la asignatura de Proyecto Hardware se
% debe explicar en este apartado el esquema del proyecto (ficheros y
% funciones que lo componen), número total de líneas de código y
% horas de dedicación. Así cómo el código desarrollado comentado.

\section{Resultados y características}
% Hay que presentar los resultados, explicarlos y analizarlos.

\section{Conclusiones}
% Es lo último que se lee, por tanto es una sección muy importante
% que se debe utilizar para remarcar los mensajes que queremos que
% el lector reciba. Por ejemplo, si estamos evaluando un producto
% podemos enfatizar sus puntos fuertes y sus puntos débiles, y
% señalar posibilidades de mejora. Normalmente al final se incluyen
% referencias, bibliografía, índice de expresiones técnicas y
% anexos.

\subsection{Margen de mejora}

\section{Bibliografía}
\begin{itemize}
\item Manuales de consulta ofrecidos por el profesorado
\item Proyectos de la placa ofrecidos por el profesorado
\item \url{http://infocenter.arm.com/help/index.jsp} (Especialmente el
  manual de referencia de ARM7)
  % Little semicolono
  % that makes me not compile this,
  % where are you missing?
\item \url{http://www.sudoku-solutions.com/}
\end{itemize}
\end{document}
