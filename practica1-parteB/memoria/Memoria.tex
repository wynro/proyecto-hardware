\documentclass[12pt,letterpaper]{article}
\usepackage[spanish]{babel}
\usepackage[utf8]{inputenc}
\usepackage[right=2cm,left=3cm,top=2cm,bottom=2cm,headsep=0cm,footskip=0.5cm]{geometry}
\usepackage[dvips]{graphicx}

\begin{document}


\section{Resumen}
%Es un apartado fundamental. Es lo primero que se lee y muchas*
%veces lo único que lee. Es la síntesis de todo el trabajo
%realizado, qué, cómo y por qué hemos hecho el trabajo. Debe ser
%auto contenido y debemos esbozar nuestras conclusiones.




\section{Introducción}
%Enmarca y sitúa el trabajo a realizar.

Sudoku es un puzle desarrollado en 1970, muy popular en Japón
desde 1986 y conocido mundialmente desde 2005.
El objetivo del sudoku es rellenar una cuadrícula de 9 x 9 celdas
(81 casillas) dividida en subcuadrículas de 3 x 3 (también llamadas
cajas o regiones)con las cifras del 1 al 9 partiendo de algunos
números ya dispuestos en algunas de las celdas. Aunque se podrían usar
colores, letras, figuras, se conviene en usar números para mayor
claridad, lo que importa, es que sean nueve elementos diferenciados,
que no se deben repetir en una misma fila, columna o subcuadrícula.
Por el propio diseño del puzle es parte primordial de su resolución 
conocer qué números pueden ser candidatos de ocupar una casilla concreta
y, por extensión, todas las demás casillas.
Por eso nuestro proyecto es desarrollar un asistente electrónico que
de forma automática sugiera los candidatos para cada celda dado
un puzle concreto.





\section{Objetivos}
%Explica qué se quiere conseguir.

Aunque el objetivo principal del proyecto sea la creación de un sencillo
programa de ayuda a la resolución del popular puzzle. Desde un punto de 
vista pedagógico existen otros objetivos no estrictamente relacionados
con el objetivo principal:

\begin{itemize}
\item Aprender a desarrollar codigo para microcontroladores en
múltiples lenguajes (tanto de alto como de bajo nivel), y
conocer las tecnicas que existen para el entrelazado y compilado
conjunto de un programa escrito en múltiples lenguajes.
\item Comparar a nivel puramente técnico múltiples lenguajes y
combinaciones de los mismos, principalmente mediante la métrica
"nº de instrucciones de procesador para terminar una tarea"
\end{itemize}





\clearpage
\section{Metodología}
%Describe los pasos realizados para llegar hasta los
%resultados. Todas aquellas decisiones de diseño tomadas en el
%proceso deben incluirse.En la asignatura de Proyecto Hardware se
%debe explicar en este apartado el esquema del proyecto (ficheros y
%funciones que lo componen), número total de líneas de código y
%horas de dedicación. Así cómo el código desarrollado comentado.

\subsection{Diseño\\ {\small Dedicación: 2 horas}}

El primer paso para afrontar el proyecto ha sido la estructuración
y división del problema para solucionar sus partes de modo ordenado
y coherente.\\
Con técnicas como la división en ficheros de las rutias y la definición
(pre-condición y post-condición) previa a la implementación hemos logrado
construir un código bien estructurado de una forma agil y sencilla
sin preocuparnos en el momento del diseño de los problemas de implemntación
que irían surgiendo.\\

\subsection{Implementación\\ {\small Dedicación: 2 horas}}

\subsection{Control de Resultados\\ {\small Dedicación: 2 horas}}

\subsection{Obtención de métricas\\ {\small Dedicación: 2 horas}}



\section{Resultados}
%Hay que presentar los resultados, explicarlos y analizarlos.

Como se puede apreciar en la tabla las métricas revelan que

% [Introducir aquí métricas y conclusión]




\section{Conclusiones}
%Es lo último que se lee, por tanto es una sección muy importante
%que se debe utilizar para remarcar los mensajes que queremos que
%el lector reciba. Por ejemplo, si estamos evaluando un producto
%podemos enfatizar sus puntos fuertes y sus puntos débiles, y
%señalar posibilidades de mejora. Normalmente al final se incluyen
%referencias, bibliografía, índice de expresiones técnicas y
%anexos.




\subsection{Anexos}



\end{document}