\documentclass[12pt,letterpaper]{article}
\usepackage[spanish]{babel}
\usepackage[utf8]{inputenc}
\usepackage[right=2cm,left=3cm,top=2cm,bottom=2cm,headsep=0cm,footskip=0.5cm]{geometry}
\usepackage[pdftex]{graphicx}
\usepackage{hyperref}
\usepackage{makeidx}
% \usepackage{verse}
% \newcommand{\attrib}[1]{%
% \nopagebreak{\raggedleft\footnotesize #1\par}}
% \renewcommand{\poemtitlefont}{\normalfont\large\itshape\centering}

\makeindex
\pagenumbering{arabic}

% Titulo Antiguo, para la posteridad, como no se usa...
\title{Proyecto Hardware - Practica 1\\Desarrollo combinado y
  optimización de programas en multiples lenguajes de alto y bajo
  nivel} \author{Guillermo Robles Gonzalez - NIP: 604409\\Sergio
  Martín Segura - NIP: 622612} \date{\today}



% It's Dangerous to Go Alone! Take This
\begin{document}
% Primera pagina (Titulo)
\begin{titlepage}
  \centering
  \includegraphics[width=0.75\textwidth]{logoUZ.png}\\
  % \vspace{1cm}
  % Mejor la imagen por favor
  % {\scshape\LARGE Universidad de Zaragoza \\}
  \vspace{0.5cm}
  {\scshape\Large Proyecto Hardware\\Practica 1\\}
  \vspace{1.5cm}
  {\huge\bfseries Desarrollo combinado y optimización de programas en
    multiples lenguajes de alto y bajo nivel\\}
  \vspace{2cm}
  {\Large \textit{Guillermo Robles González - NIP: 604409}\\}
  {\Large \textit{Sergio Martín Segura - NIP: 622612}\\}
  \vfill
  Supervisado por:\\
  Javier Resano Ezcaray (coordinador)\\
  María Villarroya Gaudó\\
  Enrique Torres Moreno\\
  Jesús Alastruey Benedé\\
  Darío Suárez Gracia\\
  % Dr.~Emmett \textsc{Doc} Lathrop Brown\\
  % Martin \textsc{Marty} Seamus McFly\\
  % \textbf{Marty:} Un momento, un momento Doc, ¿Quieres decir que has
  % construido una máquina del tiempo...? ¡¡¿En un DeLorean?!!\\
  % \textbf{Doc:} Bueno, siempre he pensado ``si vas a viajar al
  % pasado, hazlo con estilo'' ¿Y que mejor que con un DeLorean?
  \vfill
  % Bottom of the page
  {\large \today}
  % Creado con \LaTeX
\end{titlepage}
% Segunda pagina (Indice)
\tableofcontents
\clearpage
% El resto
\section{Resumen}
% Es un apartado fundamental. Es lo primero que se lee y muchas*
% veces lo único que lee. Es la síntesis de todo el trabajo
% realizado, qué, cómo y por qué hemos hecho el trabajo. Debe ser
% auto contenido y debemos esbozar nuestras conclusiones.

% [*] todas
El trabajo que se realiza es la programación de una serie de
funciones y rutinas, en varios lenguajes de bajo nivel y uno de alto
nivel, para la ayuda a la realización del conocido puzle
sudoku. Este trabajo se enmarca dentro de la asignatura de Proyecto
Hardware de 3 del grado superior en Ingeniería informática por la
Universidad de Zaragoza. Además, se comparará la eficiencia relativa
de los distintos lenguajes, y se verá la capacidad de los
compiladores modernos de realizar compilación cruzada,
permitiéndonos combinar código escrito en múltiples lenguajes.

\section{Introducción}
% Enmarca y sitúa el trabajo a realizar.
Sudoku es un puzle desarrollado en 1970, muy popular en Japón desde
1986 y conocido mundialmente desde 2005.  El objetivo del sudoku es
rellenar una cuadrícula de 9 x 9 celdas (81 casillas) dividida en
subcuadrículas de 3 x 3 (también llamadas cajas o regiones)con las
cifras del 1 al 9 partiendo de algunos números ya dispuestos en
algunas de las celdas. Aunque se podrían usar colores, letras,
figuras, se conviene en usar números para mayor claridad, lo que
importa, es que sean nueve elementos diferenciados, que no se deben
repetir en una misma fila, columna o subcuadrícula.  Por el propio
diseño del puzle es parte primordial de su resolución conocer qué
números pueden ser candidatos de ocupar una casilla concreta y, por
extensión, todas las demás casillas.  Por eso nuestro proyecto es
desarrollar un asistente electrónico que de forma automática sugiera
los candidatos para cada celda dado un puzle concreto.  En esta
primera parte, el trabajo consiste en la creación de una serie de
rutinas que, dada una cuadrícula decidirá cuales son los posibles
candidatos para cada casilla vacía.
\section{Objetivos}
% Explica qué se quiere conseguir.
Aunque el objetivo principal del proyecto sea la creación de un sencillo
programa de ayuda a la resolución del popular puzzle. Desde un punto de
vista pedagógico existen otros objetivos no estrictamente relacionados
con el objetivo principal:
\begin{itemize}
\item Aprender a desarrollar codigo para microcontroladores en
  múltiples lenguajes (tanto de alto como de bajo nivel), y
  conocer las tecnicas que existen para el entrelazado y compilado
  conjunto de un programa escrito en múltiples lenguajes.
\item Comparar a nivel puramente técnico múltiples lenguajes y
  combinaciones de los mismos, principalmente mediante la métrica
  "nº de instrucciones de procesador para terminar una tarea"
\item Realización justificada de toma de decisiones sobre el proyecto,
  tanto en aspectos de diseño como de implementación
  % So META
\item Documentar correctamente el desarrollo y características del
  producto en cuestión
\end{itemize}

\clearpage
\section{Metodología}
% Describe los pasos realizados para llegar hasta los
% resultados. Todas aquellas decisiones de diseño tomadas en el
% proceso deben incluirse.En la asignatura de Proyecto Hardware se
% debe explicar en este apartado el esquema del proyecto (ficheros y
% funciones que lo componen), número total de líneas de código y
% horas de dedicación. Así cómo el código desarrollado comentado.
Se ha usado una metodología de prototipado rápido, basada en la
creación rápida de prototipos funcionales, que se iran refinando
mediante revisiones sucesivas, hasta llegar a una versión que
abarque toda la funcionalidad deseada y cumpla los objetivos de
velocidad, eficiencia o calidad deseados. \\
% https://www.youtube.com/watch?v=7WDFUcjWARU
Esta metodología permite obtener rápidamente versiones
funcionales, ya sea para testeo o discusión con el cliente, sin
embargo sufre el defecto de que una vez establecido un algoritmo,
tiende a ser difícil cambiarlo. Se ha evitado este defecto revisando
los algoritmos usados y la forma de implementarlos con cuidado, y
realizando a veces reimplementaciones completas de ciertos trozos del
programa.
\subsection{Diseño\\ {\small Dedicación: aprox 2 horas}}
El primer paso para afrontar el proyecto ha sido la estructuración y
división del problema para solucionar sus partes de modo ordenado
y coherente.\\
Con técnicas como la división en ficheros de las rutinas y la
definición (pre-condición y post-condición) previa a la implementación
hemos logrado construir un código bien estructurado de una forma ágil
y sencilla sin preocuparnos en el momento del diseño de los problemas
de implementación que irían surgiendo.
\subsection{Implementación\\ {\small Dedicación: aprox 3 horas}}
La implementación se ha realizado conjuntamente, no realizando
reparto de tareas, ya que se ha considerado que dado el tamaño del
problema en cuestión no merecía la pena.
\subsection{Control de Resultados\\ {\small Dedicación: aprox 3 horas}}
Para el control de resultados se han utilizado 3 tablas de sudoku, una
de ellas aportada con el trabajo, y las otras obtenidas de revistas de
pasatiempos.\\
Dada la metodología usada, se realizaban pruebas constantemente,
midiéndose tanto la correctitud de los algoritmos, como la eficiencia
de los mismos.\\
El control de resultados se realizó de manera manual, con algo de
ayuda de utilidades de tratamiento de texto.
\subsection{Obtención de métricas\\ {\small Dedicación: aprox 2 horas}}
Para las mediciones de correctitud, primero se realizó y comprobó el
código en C, siendo esta nuestra implementación base, y a partir de
esa implementación se resolvieron las cuadriculas a utilizar, las
cuales fueron comprobadas manualmente. A continuación, mediante la
función de exportado de zonas de memoria de Eclipse y la utilidad diff
(que permite comparar archivos de texto o binarios y resaltar las
diferencias) se comprobaba si las implementaciones en otros lenguajes
eran correctas, comparando su salida con la salida de ejemplo.\\
Para la medición del tiempo de ejecución (medida poco útil en un
entorno real, dado que depende del sistema de prueba, pero adecuado
para comparar las distintas implementaciones) se repitio la ejecución
de las rutinas en cuestion varias veces, dado que tardaban demasiado
poco para ser medidas normalmente. Para ello se utilizó un cronómetro
externo al sistema, que se iniciaba y detenía manualmente, por ello,
se ralizan varias medidas, para evitar sesgo humano.\\
Para las mediciones de tamaño de código (tanto estático como dinámico)
se ha usado la ayuda del debugger gdb, mediante la interfaz gráfica
ofrecida por el IDE Eclipse.

\section{Resultados}
% Hay que presentar los resultados, explicarlos y analizarlos.
En la escritura de las pruebas se utiliza la notación c\_a para
indicar que la función llamadora (en nuestro caso la que recorre la
cuadrícula) esta realizada en C, y la función hoja (en nuestro
caso, la que analiza una casilla particular y marca sus candidatos)
esta en ARM; de la misma forma se usa la sigla t para funciones Thumb.\\
Todas las medidas están en segundos

Tiempos de ejecución:
\begin{itemize}
\item {\large Tiempo de ejecución}\\
  1000 pruebas:
  \begin{center}
    \begin{tabular}{ r | r | r | r | r | r | r | r }
      Función & $t_1 (s)$ & $t_2 (s)$ & $t_3 (s)$ & $t_4 (s)$ & $t_5 (s)$ & Media (s) & TPE\footnotemark[1] (s) \\ \hline
      c\_c    & 16.02 & 16.23 & 16.27 & 16.19 & 16.11 & 16.16 & 0.001616 \\
      c\_a    & 3.22  & 2.8   & 2.76  & 2.73  & 2.73  & 2.848 & 0.000285 \\
      c\_t    & 3.57  & 3.39  & 3.51  & 3.51  & 3.51  & 3.498 & 0.000349 \\ \hline
      a\_c    & 16.43 & 17.12 & 15.92 & 15.73 & 15.85 & 16.21 & 0.001621 \\
      a\_a    & 2.5   & 2.63  & 2.48  & 2.54  & 2.57  & 2.544 & 0.000254 \\
      a\_t    & 3.22  & 3.2   & 3.26  & 3.16  & 3.32  & 3.232 & 0.000323 \\ \hline
    \end{tabular}
  \end{center}
  \footnotemark[1]{Tiempo Por Ejecución de la subrutina en cuestión}

  10000 pruebas:\\
  \begin{center}
    \begin{tabular}{ r | r | r | r | r | r | r | r }
      Función & $t_1 (s)$ & $t_2 (s)$ & $t_3 (s)$ & $t_4 (s)$ & $t_5 (s)$ & Media (s) & TPE\footnotemark[1] (s) \\ \hline
      c\_c    & 155.12 & 152.56 & 157.72 & 158.64 & 158.64 & 156.54 & 0.001565 \\
      c\_a    & 23.26  & 23.13  & 23.20  & 23.19  & 23.19  & 23.194 & 0.000232 \\
      c\_t    & 31.53  & 31.41  & 31.98  & 31.36  & 31.36  & 31.528 & 0.000315 \\ \hline
      a\_c    & 155.26 & 155.26 & 156.32 & 156.69 & 156.69 & 156.69 & 0.001567 \\
      a\_a    & 21.86  & 22.32  & 22.07  & 22.05  & 22.05  & 22.07  & 0.000221 \\
      a\_t    & 27.41  & 28.08  & 29.04  & 28.95  & 28.95  & 28.486 & 0.000285 \\ \hline
    \end{tabular}
  \end{center}
  \footnotemark[1]{Tiempo Por Ejecución de la subrutina en cuestión}
  \clearpage
\item {\large Tamaño de código}\\
  En la comparación se ha medido la función candidatos, que a partir
  de una casilla obtiene los números que pueden ir allí. Se ha usado
  como base la casilla (0,2) del sudoku dado como ejemplo, dado que es
  la primera casilla vacía.
  \begin{center}
    \begin{tabular}{ l | r | r }
      % TODO: terminar
      & \multicolumn{2}{|c}{Tamaño} \\
      Función                                 & estático (bytes)   & dinámico (instrucciones) \\ \hline
      sudoku\_candidatos\_c                   & 676                  & 2634                       \\
      sudoku\_candidatos\_arm                 & 288                  & 349                        \\
      sudoku\_candidatos\_thumb (Con prólogo) & 254                  & 480                        \\
      sudoku\_candidatos\_thumb (Sin prólogo) & 214                  & 489                        \\ \hline
    \end{tabular}
  \end{center}
  En este caso se han medido los tamaños de código de las funciones
  que calculan la tabla de sudoku completa; ha sido imposible la
  medición del tamaño dinámico por razones de coste computacional
  \begin{center}
    \begin{tabular}{ l | r | r }
      % TODO: terminar
      Función                                   & Tamaño estático (bytes) \\ \hline
      sudoku\_recalcular\_c\_c                  & 172                       \\
      sudoku\_recalcular\_c\_a                  & 172                       \\
      sudoku\_recalcular\_c\_t \footnotemark[1] & 172                       \\
      sudoku\_recalcular\_a\_c                  & 80                        \\
      sudoku\_recalcular\_a\_a                  & 80                        \\
      sudoku\_recalcular\_a\_t                  & 100                       \\
    \end{tabular}
  \end{center}
  \footnotemark{Es de notar que esta función no llama a
    sudoku\_candidatos\_thumb directamente, dada la imposibilidad de
    llamar a funciones en thumb desde C; sino a una función especial
    sudoku\_candidatos\_thumb\_prologo, que está escrita en ARM y
    ejecuta la función thumb deseada}
\end{itemize}

% \subsection{Conclusiones de las métricas}

% [ Introducir aquí métricas y conclusión]
%                               @B@B@
%                               @@B@B
%                               @B@B@
%                               B@B@@
%                               @@@B@
%                               B@B@B
%                               @@@@@
%                               B@B@@
%                               @B@B@
%                               B@@@B
%                               @@@B@
%                               B@@@B
%                               @B@B@
%                               B@B@B
%                               @@@B@
%                             B@B@B@B@B,
%                             MqJ7i7YPM.
%                                7@:
%                               r@B@:
%                              r@@@B@:
%                             ;@B@B@B@:
%                            r@B@B@B@@@:
%                           ;@@@B@B@B@B@:
%                          i@B@@@B@B@B@B@:
%                         i@B@B@B@B@B@@@B@:
%                        i@B@B@B@@@B@B@B@B@:
%                       i@B@B@B@B@B@B@@@B@B@:
%                      i@B@B@B@u. , .u@B@@@B@:
%                     i@B@B@Br   .@    7@@B@@@:
%                    i@B@B@u     u@.     0@B@B@:
%                   i@@@B@       XB:      .@B@B@:
%                  :@B@B@M       F@:       B@B@@@,
%                 :@B@B@B@Bv     LB      5B@B@B@B@,
%                :@B@B@B@B@@@:    @    :@B@B@B@B@@@,
%               :@B@B@B@B@@@@@Bv     v@@B@B@B@B@B@B@,
%              :@B@@@B@B@B@B@B@B@B@B@@@B@B@@@B@B@B@B@,
%             :@B@B@B@B@B@Br7@@@@@B@B@BB:5B@@@B@B@B@B@,
%            :@B@B@B@B@B@B@    j@B@BBr   J@B@@@B@B@B@B@,
%           :@B@B@B@B@B@B@B:      r      S@@B@@@B@B@B@B@,
%          ,@B@B@B@B@B@@@B@.    7G@Pi    U@B@B@B@B@B@@@B@.
%         ,@B@B@B@B@@@B@B@B. YB@B@B@B@B; LB@B@@@@@B@@@@@@@.
%        ,@@@@@B@@@B@B@B@B@B@B@@@@@B@@@B@@@B@@@@@B@B@B@B@B@.
%       ,@B@B@B@@@B@B@@@B@@@B@B@B@B@@@B@B@B@B@B@@@B@B@B@B@B@.
%      .@@@B@B@B@B@@@B@B@B@B@B@B@B@B@@@B@B@B@B@B@@@@@B@B@B@B@.
%     ,@B@@@B@B@@@B@@@B@B@B@B@B@B@B@@@B@@@B@@@B@@@B@B@B@B@B@B@.
%    .@B@B@B@B@B@B@B@B@B@B@B@B@B@B@B@B@@@B@B@@@B@B@B@B@B@@@B@B@.
%   ,@B@B@B@B@B@B@@@B@B@B@@@B@B@B@B@B@B@@@B@B@B@B@B@B@@@B@B@B@B@.
%  :@B@@@B@B@B@B@B@B@@@B@@@B@B@B@B@B@B@B@B@B@B@B@B@B@@@B@B@@@B@@@.
% i@B@B@BBB@MBB@B@B@B@MMB@O@B@B@B@B@B@B@B@BBM@MBB@BMOM8GBMZOOMMBB@
%         .5            Yu                   B          k
%         rP            15                   @          P7
%         v0            LB    @i             B          FU
%         7G             @   MBi            U@          Uq
%         vE             v@YM @i           J@           vG
%         vG               :  Bi       MMXE2            rO
%         LN                  @v       B7               rB
%         i8                  i:       @j               :@
%         .B                           BN               ,B
%          @                           @q               iM
%          B7                          Bi          r    u5
%          iB  .v                                  78 0 B:
%           @r iL Z                               8iuqO,B
%            Mi@kO:                                ::uO7

\section{Conclusiones}
% Es lo último que se lee, por tanto es una sección muy importante
% que se debe utilizar para remarcar los mensajes que queremos que
% el lector reciba. Por ejemplo, si estamos evaluando un producto
% podemos enfatizar sus puntos fuertes y sus puntos débiles, y
% señalar posibilidades de mejora. Normalmente al final se incluyen
% referencias, bibliografía, índice de expresiones técnicas y
% anexos.

La primera observación que se hace es la inmensa diferencia en
eficiencia entre C y ensamblador, incluso en algoritmos sencillos como
este, sin embargo la diferencia de dificultad al realizar los
programas es considerable. Es de notar que la principal diferencia es
en la función candidatos, que es la que más carga aporta al programa;
también puede verse como el llamador es poco importante, obteniéndose
diferencias de apenas unas pocas cienmilésimas de segundo entre las
llamadas a c\_a y a\_a.\\
La elección de C como función hoja es impensable, dado que se obtienen
tiempos del orden de 6-7 veces mayores, una posibilidad sería utilizar
una compilación mas agresiva en la eficiencia, que aunque dificulta la
comprensión del código generado, generalmente es más eficiente. En la
situación en la que estamos (Interacción con un humano) la diferencia
de eficiencia entre ASM y Thumb es despreciable, por lo tanto el
factor que determinará la elección será el tamaño de la memoria
disponible, escogiéndose Thumb en caso de que esta sea baja, y ARM en
caso contrario.\\
En conclusión, se observa que la forma que balancea mejor comodidad de
desarrollo y eficiencia es la programación del cuerpo principal de la
aplicación en un lenguaje de alto nivel (C en nuestro caso), y la
programación de las subrutinas especialmente costosas en un lenguaje
de bajo nivel (ensamblador).

\subsection{Margen de mejora}
Para la tarea para la cual se va a utilizar (un sistema que incluye
interacción con un ser humano) la velocidad de las subrutinas es
suficientemente alta, sin embargo, si comparamos nuestra
implementación con implementaciones de algunos compañeros, vemos que
aún queda muchas posibilidades de mejorar en ese aspecto.
\clearpage
\section{Bibliografía}
\begin{itemize}
\item Manuales de consulta ofrecidos por el profesorado
\item \url{https://es.wikipedia.org/wiki/Sudoku}
\item \url{http://infocenter.arm.com/help/index.jsp} (Especialmente el
  manual de referencia de ARM7)
  % Little semicolono
  % that makes me not compile this,
  % where are you missing?
\item \url{http://www.arm.com/products/processors/classic/arm7/index.php}
\item \url{http://www.sudoku-solutions.com/}
\end{itemize}
\end{document}
