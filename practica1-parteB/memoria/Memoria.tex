\documentclass[12pt,letterpaper]{article}
\usepackage[spanish]{babel}
\usepackage[utf8]{inputenc}
\usepackage[right=2cm,left=3cm,top=2cm,bottom=2cm,headsep=0cm,footskip=0.5cm]{geometry}
\usepackage[pdftex]{graphicx}
\usepackage{hyperref}
\usepackage{makeidx}
% \usepackage{verse}
% \newcommand{\attrib}[1]{%
% \nopagebreak{\raggedleft\footnotesize #1\par}}
% \renewcommand{\poemtitlefont}{\normalfont\large\itshape\centering}

\makeindex
\pagenumbering{arabic}

% Titulo Antiguo, para la posteridad, como no se usa...
\title{Proyecto Hardware - Practica 1\\Desarrollo combinado y
  optimizaci\'on de programas en multiples lenguajes de alto y bajo
  nivel} \author{Guillermo Robles Gonzalez - NIP: 604409\\Sergio
  Mart\'in Segura - NIP: 622612} \date{\today}



% It's Dangerous to Go Alone! Take This
\begin{document}
% Primera pagina (Titulo)
\begin{titlepage}
  \centering
  \includegraphics[width=0.75\textwidth]{logoUZ.png}\\
  % \vspace{1cm}
  % Mejor la imagen por favor
  % {\scshape\LARGE Universidad de Zaragoza \\}
  \vspace{0.5cm}
  {\scshape\Large Proyecto Hardware\\Practica 1\\}
  \vspace{1.5cm}
  {\huge\bfseries Desarrollo combinado y optimizaci\'on de programas en
    multiples lenguajes de alto y bajo nivel\\}
  \vspace{2cm}
  {\Large \textit{Guillermo Robles Gonz\'alez - NIP: 604409}\\}
  {\Large \textit{Sergio Mart\'in Segura - NIP: 622612}\\}
  \vfill
  Supervisado por:\\
  Javier Resano Ezcaray (coordinador)\\
  Mar\'ia Villarroya Gaud\'o\\
  Enrique Torres Moreno\\
  Jes\'us Alastruey Bened\'e\\
  Dar\'io Su\'arez Gracia\\
  % Dr.~Emmett \textsc{Doc} Lathrop Brown\\
  % Martin \textsc{Marty} Seamus McFly\\
  % \textbf{Marty:} Un momento, un momento Doc, ¿Quieres decir que has
  % construido una m\'aquina del tiempo...? ¡¡¿En un DeLorean?!!\\
  % \textbf{Doc:} Bueno, siempre he pensado ``si vas a viajar al
  % pasado, hazlo con estilo'' ¿Y que mejor que con un DeLorean?
  \vfill
  % Bottom of the page
  {\large \today}
  % Creado con \LaTeX
\end{titlepage}
% Segunda pagina (Indice)
\tableofcontents
\clearpage
% El resto
\section{Resumen}
% Es un apartado fundamental. Es lo primero que se lee y muchas*
% veces lo \'unico que lee. Es la s\'intesis de todo el trabajo
% realizado, qu\'e, c\'omo y por qu\'e hemos hecho el trabajo. Debe ser
% auto contenido y debemos esbozar nuestras conclusiones.

% [*] todas
El trabajo que se realiza es la programaci\'on de una serie de
funciones y rutinas, en varios lenguajes de bajo nivel y uno de alto
nivel, para la ayuda a la realizaci\'on del conocido puzle
sudoku. Este trabajo se enmarca dentro de la asignatura de Proyecto
Hardware de 3 del grado superior en Ingenier\'ia inform\'atica por la
Universidad de Zaragoza. Adem\'as, se comparar\'a la eficiencia relativa
de los distintos lenguajes, y se ver\'a la capacidad de los
compiladores modernos de realizar compilaci\'on cruzada,
permiti\'endonos combinar c\'odigo escrito en m\'ultiples lenguajes.

\section{Introducci\'on}
% Enmarca y sit\'ua el trabajo a realizar.
Sudoku es un puzle desarrollado en 1970, muy popular en Jap\'on desde
1986 y conocido mundialmente desde 2005.  El objetivo del sudoku es
rellenar una cuadr\'icula de 9 x 9 celdas (81 casillas) dividida en
subcuadr\'iculas de 3 x 3 (tambi\'en llamadas cajas o regiones)con las
cifras del 1 al 9 partiendo de algunos n\'umeros ya dispuestos en
algunas de las celdas. Aunque se podr\'ian usar colores, letras,
figuras, se conviene en usar n\'umeros para mayor claridad, lo que
importa, es que sean nueve elementos diferenciados, que no se deben
repetir en una misma fila, columna o subcuadr\'icula.  Por el propio
diseño del puzle es parte primordial de su resoluci\'on conocer qu\'e
n\'umeros pueden ser candidatos de ocupar una casilla concreta y, por
extensi\'on, todas las dem\'as casillas.  Por eso nuestro proyecto es
desarrollar un asistente electr\'onico que de forma autom\'atica sugiera
los candidatos para cada celda dado un puzle concreto.  En esta
primera parte, el trabajo consiste en la creaci\'on de una serie de
rutinas que, dada una cuadr\'icula decidir\'a cuales son los posibles
candidatos para cada casilla vac\'ia.
\section{Objetivos}
% Explica qu\'e se quiere conseguir.
Aunque el objetivo principal del proyecto sea la creaci\'on de un sencillo
programa de ayuda a la resoluci\'on del popular puzzle. Desde un punto de
vista pedag\'ogico existen otros objetivos no estrictamente relacionados
con el objetivo principal:
\begin{itemize}
\item Aprender a desarrollar codigo para microcontroladores en
  m\'ultiples lenguajes (tanto de alto como de bajo nivel), y
  conocer las tecnicas que existen para el entrelazado y compilado
  conjunto de un programa escrito en m\'ultiples lenguajes.
\item Comparar a nivel puramente t\'ecnico m\'ultiples lenguajes y
  combinaciones de los mismos, principalmente mediante la m\'etrica
  "nº de instrucciones de procesador para terminar una tarea"
\item Realizaci\'on justificada de toma de decisiones sobre el proyecto,
  tanto en aspectos de diseño como de implementaci\'on
  % So META
\item Documentar correctamente el desarrollo y caracter\'isticas del
  producto en cuesti\'on
\end{itemize}

\clearpage
\section{Metodolog\'ia}
% Describe los pasos realizados para llegar hasta los
% resultados. Todas aquellas decisiones de diseño tomadas en el
% proceso deben incluirse.En la asignatura de Proyecto Hardware se
% debe explicar en este apartado el esquema del proyecto (ficheros y
% funciones que lo componen), n\'umero total de l\'ineas de c\'odigo y
% horas de dedicaci\'on. As\'i c\'omo el c\'odigo desarrollado comentado.
Se ha usado una metodolog\'ia de prototipado r\'apido, basada en la
creaci\'on r\'apida de prototipos funcionales, que se iran refinando
mediante revisiones sucesivas, hasta llegar a una versi\'on que
abarque toda la funcionalidad deseada y cumpla los objetivos de
velocidad, eficiencia o calidad deseados. \\
% https://www.youtube.com/watch?v=7WDFUcjWARU
Esta metodolog\'ia permite obtener r\'apidamente versiones
funcionales, ya sea para testeo o discusi\'on con el cliente, sin
embargo sufre el defecto de que una vez establecido un algoritmo,
tiende a ser dif\'icil cambiarlo. Se ha evitado este defecto revisando
los algoritmos usados y la forma de implementarlos con cuidado, y
realizando a veces reimplementaciones completas de ciertos trozos del
programa.
\subsection{Diseño\\ {\small Dedicaci\'on: aprox 2 horas}}
El primer paso para afrontar el proyecto ha sido la estructuraci\'on y
divisi\'on del problema para solucionar sus partes de modo ordenado
y coherente.\\
Con t\'ecnicas como la divisi\'on en ficheros de las rutinas y la
definici\'on (pre-condici\'on y post-condici\'on) previa a la implementaci\'on
hemos logrado construir un c\'odigo bien estructurado de una forma \'agil
y sencilla sin preocuparnos en el momento del diseño de los problemas
de implementaci\'on que ir\'ian surgiendo.
\subsection{Implementaci\'on\\ {\small Dedicaci\'on: aprox 3 horas}}
La implementaci\'on se ha realizado conjuntamente, no realizando
reparto de tareas, ya que se ha considerado que dado el tama\~no del
problema en cuesti\'on no merec\'ia la pena.
\subsection{Control de Resultados\\ {\small Dedicaci\'on: aprox 3 horas}}
Para el control de resultados se han utilizado 3 tablas de sudoku, una
de ellas aportada con el trabajo, y las otras obtenidas de revistas de
pasatiempos.\\
Dada la metodolog\'ia usada, se realizaban pruebas constantemente,
midi\'endose tanto la correctitud de los algoritmos, como la eficiencia
de los mismos.\\
El control de resultados se realiz\'o de manera manual, con algo de
ayuda de utilidades de tratamiento de texto.
\subsection{Obtenci\'on de m\'etricas\\ {\small Dedicaci\'on: aprox 2 horas}}
Para las mediciones de correctitud, primero se realiz\'o y comprob\'o el
c\'odigo en C, siendo esta nuestra implementaci\'on base, y a partir de
esa implementaci\'on se resolvieron las cuadriculas a utilizar, las
cuales fueron comprobadas manualmente. A continuaci\'on, mediante la
funci\'on de exportado de zonas de memoria de Eclipse y la utilidad diff
(que permite comparar archivos de texto o binarios y resaltar las
diferencias) se comprobaba si las implementaciones en otros lenguajes
eran correctas, comparando su salida con la salida de ejemplo.\\
Para la medici\'on del tiempo de ejecuci\'on (medida poco \'util en un
entorno real, dado que depende del sistema de prueba, pero adecuado
para comparar las distintas implementaciones) se repitio la ejecuci\'on
de las rutinas en cuestion varias veces, dado que tardaban demasiado
poco para ser medidas normalmente. Para ello se utiliz\'o un cron\'ometro
externo al sistema, que se iniciaba y deten\'ia manualmente, por ello,
se ralizan varias medidas, para evitar sesgo humano.\\
Para las mediciones de tamaño de c\'odigo (tanto est\'atico como din\'amico)
se ha usado la ayuda del debugger gdb, mediante la interfaz gr\'afica
ofrecida por el IDE Eclipse.

\section{Resultados}
% Hay que presentar los resultados, explicarlos y analizarlos.
En la escritura de las pruebas se utiliza la notaci\'on c\_a para
indicar que la funci\'on llamadora (en nuestro caso la que recorre la
cuadr\'icula) esta realizada en C, y la funci\'on hoja (en nuestro
caso, la que analiza una casilla particular y marca sus candidatos)
esta en ARM; de la misma forma se usa la sigla t para funciones Thumb.\\
Todas las medidas est\'an en segundos

Tiempos de ejecuci\'on:
\begin{itemize}
\item {\large Tiempo de ejecuci\'on}\\
  1000 pruebas:
  \begin{center}
    \begin{tabular}{ r | r | r | r | r | r | r | r }
      Funci\'on & $t_1 (s)$ & $t_2 (s)$ & $t_3 (s)$ & $t_4 (s)$ & $t_5 (s)$ & Media (s) & TPE\footnotemark[1] (s) \\ \hline
      c\_c    & 16.02 & 16.23 & 16.27 & 16.19 & 16.11 & 16.16 & 0.001616 \\
      c\_a    & 3.22  & 2.8   & 2.76  & 2.73  & 2.73  & 2.848 & 0.000285 \\
      c\_t    & 3.57  & 3.39  & 3.51  & 3.51  & 3.51  & 3.498 & 0.000349 \\ \hline
      a\_c    & 16.43 & 17.12 & 15.92 & 15.73 & 15.85 & 16.21 & 0.001621 \\
      a\_a    & 2.5   & 2.63  & 2.48  & 2.54  & 2.57  & 2.544 & 0.000254 \\
      a\_t    & 3.22  & 3.2   & 3.26  & 3.16  & 3.32  & 3.232 & 0.000323 \\ \hline
    \end{tabular}
  \end{center}
  \footnotemark[1]{Tiempo Por Ejecuci\'on de la subrutina en cuesti\'on}

  10000 pruebas:\\
  \begin{center}
    \begin{tabular}{ r | r | r | r | r | r | r | r }
      Funci\'on & $t_1 (s)$ & $t_2 (s)$ & $t_3 (s)$ & $t_4 (s)$ & $t_5 (s)$ & Media (s) & TPE\footnotemark[1] (s) \\ \hline
      c\_c    & 155.12 & 152.56 & 157.72 & 158.64 & 158.64 & 156.54 & 0.001565 \\
      c\_a    & 23.26  & 23.13  & 23.20  & 23.19  & 23.19  & 23.194 & 0.000232 \\
      c\_t    & 31.53  & 31.41  & 31.98  & 31.36  & 31.36  & 31.528 & 0.000315 \\ \hline
      a\_c    & 155.26 & 155.26 & 156.32 & 156.69 & 156.69 & 156.69 & 0.001567 \\
      a\_a    & 21.86  & 22.32  & 22.07  & 22.05  & 22.05  & 22.07  & 0.000221 \\
      a\_t    & 27.41  & 28.08  & 29.04  & 28.95  & 28.95  & 28.486 & 0.000285 \\ \hline
    \end{tabular}
  \end{center}
  \footnotemark[1]{Tiempo Por Ejecuci\'on de la subrutina en cuesti\'on}
  \clearpage
\item {\large Tamaño de c\'odigo}\\
  En la comparaci\'on se ha medido la funci\'on candidatos, que a partir
  de una casilla obtiene los n\'umeros que pueden ir all\'i. Se ha usado
  como base la casilla (0,2) del sudoku dado como ejemplo, dado que es
  la primera casilla vac\'ia.
  \begin{center}
    \begin{tabular}{ l | r | r }
      % TODO: terminar
      & \multicolumn{2}{|c|}{Tamano} \\
      Funci\'on                                 & est\'atico (bytes)   & Tamaño din\'amico (instrucciones) \\ \hline
      sudoku\_candidatos\_c                   & 676                       & 2634                            \\
      sudoku\_candidatos\_arm                 & 288                       & 349                             \\
      sudoku\_candidatos\_thumb (Con pr\'ologo) & 254                       & 480                             \\
      sudoku\_candidatos\_thumb (Sin pr\'ologo) & 214                       & 489                             \\ \hline
    \end{tabular}
  \end{center}
  En este caso se han medido los tamaños de c\'odigo de las funciones
  que calculan la tabla de sudoku completa; ha sido imposible la
  medici\'on del tamaño din\'amico por razones de coste computacional
  \begin{center}
    \begin{tabular}{ l | r | r }
      % TODO: terminar
      Funci\'on                                   & Tamaño est\'atico (bytes) \\ \hline
      sudoku\_recalcular\_c\_c                  & 172                       \\
      sudoku\_recalcular\_c\_a                  & 172                       \\
      sudoku\_recalcular\_c\_t \footnotemark[1] & 172                       \\
      sudoku\_recalcular\_a\_c                  & 80                        \\
      sudoku\_recalcular\_a\_a                  & 80                        \\
      sudoku\_recalcular\_a\_t                  & 100                       \\
    \end{tabular}
  \end{center}
  \footnotemark{Es de notar que esta funci\'on no llama a
    sudoku\_candidatos\_thumb directamente, dada la imposibilidad de
    llamar a funciones en thumb desde C; sino a una funci\'on especial
    sudoku\_candidatos\_thumb\_prologo, que est\'a escrita en ARM y
    ejecuta la funci\'on thumb deseada}
\end{itemize}

% \subsection{Conclusiones de las m\'etricas}

% [ Introducir aqu\'i m\'etricas y conclusi\'on]
%                               @B@B@
%                               @@B@B
%                               @B@B@
%                               B@B@@
%                               @@@B@
%                               B@B@B
%                               @@@@@
%                               B@B@@
%                               @B@B@
%                               B@@@B
%                               @@@B@
%                               B@@@B
%                               @B@B@
%                               B@B@B
%                               @@@B@
%                             B@B@B@B@B,
%                             MqJ7i7YPM.
%                                7@:
%                               r@B@:
%                              r@@@B@:
%                             ;@B@B@B@:
%                            r@B@B@B@@@:
%                           ;@@@B@B@B@B@:
%                          i@B@@@B@B@B@B@:
%                         i@B@B@B@B@B@@@B@:
%                        i@B@B@B@@@B@B@B@B@:
%                       i@B@B@B@B@B@B@@@B@B@:
%                      i@B@B@B@u. , .u@B@@@B@:
%                     i@B@B@Br   .@    7@@B@@@:
%                    i@B@B@u     u@.     0@B@B@:
%                   i@@@B@       XB:      .@B@B@:
%                  :@B@B@M       F@:       B@B@@@,
%                 :@B@B@B@Bv     LB      5B@B@B@B@,
%                :@B@B@B@B@@@:    @    :@B@B@B@B@@@,
%               :@B@B@B@B@@@@@Bv     v@@B@B@B@B@B@B@,
%              :@B@@@B@B@B@B@B@B@B@B@@@B@B@@@B@B@B@B@,
%             :@B@B@B@B@B@Br7@@@@@B@B@BB:5B@@@B@B@B@B@,
%            :@B@B@B@B@B@B@    j@B@BBr   J@B@@@B@B@B@B@,
%           :@B@B@B@B@B@B@B:      r      S@@B@@@B@B@B@B@,
%          ,@B@B@B@B@B@@@B@.    7G@Pi    U@B@B@B@B@B@@@B@.
%         ,@B@B@B@B@@@B@B@B. YB@B@B@B@B; LB@B@@@@@B@@@@@@@.
%        ,@@@@@B@@@B@B@B@B@B@B@@@@@B@@@B@@@B@@@@@B@B@B@B@B@.
%       ,@B@B@B@@@B@B@@@B@@@B@B@B@B@@@B@B@B@B@B@@@B@B@B@B@B@.
%      .@@@B@B@B@B@@@B@B@B@B@B@B@B@B@@@B@B@B@B@B@@@@@B@B@B@B@.
%     ,@B@@@B@B@@@B@@@B@B@B@B@B@B@B@@@B@@@B@@@B@@@B@B@B@B@B@B@.
%    .@B@B@B@B@B@B@B@B@B@B@B@B@B@B@B@B@@@B@B@@@B@B@B@B@B@@@B@B@.
%   ,@B@B@B@B@B@B@@@B@B@B@@@B@B@B@B@B@B@@@B@B@B@B@B@B@@@B@B@B@B@.
%  :@B@@@B@B@B@B@B@B@@@B@@@B@B@B@B@B@B@B@B@B@B@B@B@B@@@B@B@@@B@@@.
% i@B@B@BBB@MBB@B@B@B@MMB@O@B@B@B@B@B@B@B@BBM@MBB@BMOM8GBMZOOMMBB@
%         .5            Yu                   B          k
%         rP            15                   @          P7
%         v0            LB    @i             B          FU
%         7G             @   MBi            U@          Uq
%         vE             v@YM @i           J@           vG
%         vG               :  Bi       MMXE2            rO
%         LN                  @v       B7               rB
%         i8                  i:       @j               :@
%         .B                           BN               ,B
%          @                           @q               iM
%          B7                          Bi          r    u5
%          iB  .v                                  78 0 B:
%           @r iL Z                               8iuqO,B
%            Mi@kO:                                ::uO7

\section{Conclusiones}
% Es lo \'ultimo que se lee, por tanto es una secci\'on muy importante
% que se debe utilizar para remarcar los mensajes que queremos que
% el lector reciba. Por ejemplo, si estamos evaluando un producto
% podemos enfatizar sus puntos fuertes y sus puntos d\'ebiles, y
% señalar posibilidades de mejora. Normalmente al final se incluyen
% referencias, bibliograf\'ia, \'indice de expresiones t\'ecnicas y
% anexos.

La primera observaci\'on que se hace es la inmensa diferencia en
eficiencia entre C y ensamblador, incluso en algoritmos sencillos como
este, sin embargo la diferencia de dificultad al realizar los
programas es considerable. Es de notar que la principal diferencia es
en la funci\'on candidatos, que es la que m\'as carga aporta al programa;
tambi\'en puede verse como el llamador es poco importante, obteni\'endose
diferencias de apenas unas pocas cienmil\'esimas de segundo entre las
llamadas a c\_a y a\_a.\\
La elecci\'on de C como funci\'on hoja es impensable, dado que se obtienen
tiempos del orden de 6-7 veces mayores, una posibilidad ser\'ia utilizar
una compilaci\'on mas agresiva en la eficiencia, que aunque dificulta la
comprensi\'on del c\'odigo generado, generalmente es m\'as eficiente. En la
situaci\'on en la que estamos (Interacci\'on con un humano) la diferencia
de eficiencia entre ASM y Thumb es despreciable, por lo tanto el
factor que determinar\'a la elecci\'on ser\'a el tamaño de la memoria
disponible, escogi\'endose Thumb en caso de que esta sea baja, y ARM en
caso contrario.\\
En conclusi\'on, se observa que la forma que balancea mejor comodidad de
desarrollo y eficiencia es la programaci\'on del cuerpo principal de la
aplicaci\'on en un lenguaje de alto nivel (C en nuestro caso), y la
programaci\'on de las subrutinas especialmente costosas en un lenguaje
de bajo nivel (ensamblador).

\subsection{Margen de mejora}
Para la tarea para la cual se va a utilizar (un sistema que incluye
interacci\'on con un ser humano) la velocidad de las subrutinas es
suficientemente alta, sin embargo, si comparamos nuestra
implementaci\'on con implementaciones de algunos compañeros, vemos que
a\'un queda muchas posibilidades de mejorar en ese aspecto.
\clearpage
\section{Bibliograf\'ia}
\begin{itemize}
\item Manuales de consulta ofrecidos por el profesorado
\item \url{https://es.wikipedia.org/wiki/Sudoku}
\item \url{http://infocenter.arm.com/help/index.jsp} (Especialmente el
  manual de referencia de ARM7)
  % Little semicolono
  % that makes me not compile this,
  % where are you missing?
\item \url{http://www.arm.com/products/processors/classic/arm7/index.php}
\item \url{http://www.sudoku-solutions.com/}
\end{itemize}
\end{document}